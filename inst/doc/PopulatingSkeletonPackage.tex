% Options for packages loaded elsewhere
\PassOptionsToPackage{unicode}{hyperref}
\PassOptionsToPackage{hyphens}{url}
%
\documentclass[
]{article}
\usepackage{lmodern}
\usepackage{amssymb,amsmath}
\usepackage{ifxetex,ifluatex}
\ifnum 0\ifxetex 1\fi\ifluatex 1\fi=0 % if pdftex
  \usepackage[T1]{fontenc}
  \usepackage[utf8]{inputenc}
  \usepackage{textcomp} % provide euro and other symbols
\else % if luatex or xetex
  \usepackage{unicode-math}
  \defaultfontfeatures{Scale=MatchLowercase}
  \defaultfontfeatures[\rmfamily]{Ligatures=TeX,Scale=1}
\fi
% Use upquote if available, for straight quotes in verbatim environments
\IfFileExists{upquote.sty}{\usepackage{upquote}}{}
\IfFileExists{microtype.sty}{% use microtype if available
  \usepackage[]{microtype}
  \UseMicrotypeSet[protrusion]{basicmath} % disable protrusion for tt fonts
}{}
\makeatletter
\@ifundefined{KOMAClassName}{% if non-KOMA class
  \IfFileExists{parskip.sty}{%
    \usepackage{parskip}
  }{% else
    \setlength{\parindent}{0pt}
    \setlength{\parskip}{6pt plus 2pt minus 1pt}}
}{% if KOMA class
  \KOMAoptions{parskip=half}}
\makeatother
\usepackage{xcolor}
\IfFileExists{xurl.sty}{\usepackage{xurl}}{} % add URL line breaks if available
\IfFileExists{bookmark.sty}{\usepackage{bookmark}}{\usepackage{hyperref}}
\hypersetup{
  pdftitle={Populating the study package},
  pdfauthor={Jenna M. Reps},
  hidelinks,
  pdfcreator={LaTeX via pandoc}}
\urlstyle{same} % disable monospaced font for URLs
\usepackage[margin=1in]{geometry}
\usepackage{color}
\usepackage{fancyvrb}
\newcommand{\VerbBar}{|}
\newcommand{\VERB}{\Verb[commandchars=\\\{\}]}
\DefineVerbatimEnvironment{Highlighting}{Verbatim}{commandchars=\\\{\}}
% Add ',fontsize=\small' for more characters per line
\usepackage{framed}
\definecolor{shadecolor}{RGB}{248,248,248}
\newenvironment{Shaded}{\begin{snugshade}}{\end{snugshade}}
\newcommand{\AlertTok}[1]{\textcolor[rgb]{0.94,0.16,0.16}{#1}}
\newcommand{\AnnotationTok}[1]{\textcolor[rgb]{0.56,0.35,0.01}{\textbf{\textit{#1}}}}
\newcommand{\AttributeTok}[1]{\textcolor[rgb]{0.77,0.63,0.00}{#1}}
\newcommand{\BaseNTok}[1]{\textcolor[rgb]{0.00,0.00,0.81}{#1}}
\newcommand{\BuiltInTok}[1]{#1}
\newcommand{\CharTok}[1]{\textcolor[rgb]{0.31,0.60,0.02}{#1}}
\newcommand{\CommentTok}[1]{\textcolor[rgb]{0.56,0.35,0.01}{\textit{#1}}}
\newcommand{\CommentVarTok}[1]{\textcolor[rgb]{0.56,0.35,0.01}{\textbf{\textit{#1}}}}
\newcommand{\ConstantTok}[1]{\textcolor[rgb]{0.00,0.00,0.00}{#1}}
\newcommand{\ControlFlowTok}[1]{\textcolor[rgb]{0.13,0.29,0.53}{\textbf{#1}}}
\newcommand{\DataTypeTok}[1]{\textcolor[rgb]{0.13,0.29,0.53}{#1}}
\newcommand{\DecValTok}[1]{\textcolor[rgb]{0.00,0.00,0.81}{#1}}
\newcommand{\DocumentationTok}[1]{\textcolor[rgb]{0.56,0.35,0.01}{\textbf{\textit{#1}}}}
\newcommand{\ErrorTok}[1]{\textcolor[rgb]{0.64,0.00,0.00}{\textbf{#1}}}
\newcommand{\ExtensionTok}[1]{#1}
\newcommand{\FloatTok}[1]{\textcolor[rgb]{0.00,0.00,0.81}{#1}}
\newcommand{\FunctionTok}[1]{\textcolor[rgb]{0.00,0.00,0.00}{#1}}
\newcommand{\ImportTok}[1]{#1}
\newcommand{\InformationTok}[1]{\textcolor[rgb]{0.56,0.35,0.01}{\textbf{\textit{#1}}}}
\newcommand{\KeywordTok}[1]{\textcolor[rgb]{0.13,0.29,0.53}{\textbf{#1}}}
\newcommand{\NormalTok}[1]{#1}
\newcommand{\OperatorTok}[1]{\textcolor[rgb]{0.81,0.36,0.00}{\textbf{#1}}}
\newcommand{\OtherTok}[1]{\textcolor[rgb]{0.56,0.35,0.01}{#1}}
\newcommand{\PreprocessorTok}[1]{\textcolor[rgb]{0.56,0.35,0.01}{\textit{#1}}}
\newcommand{\RegionMarkerTok}[1]{#1}
\newcommand{\SpecialCharTok}[1]{\textcolor[rgb]{0.00,0.00,0.00}{#1}}
\newcommand{\SpecialStringTok}[1]{\textcolor[rgb]{0.31,0.60,0.02}{#1}}
\newcommand{\StringTok}[1]{\textcolor[rgb]{0.31,0.60,0.02}{#1}}
\newcommand{\VariableTok}[1]{\textcolor[rgb]{0.00,0.00,0.00}{#1}}
\newcommand{\VerbatimStringTok}[1]{\textcolor[rgb]{0.31,0.60,0.02}{#1}}
\newcommand{\WarningTok}[1]{\textcolor[rgb]{0.56,0.35,0.01}{\textbf{\textit{#1}}}}
\usepackage{graphicx,grffile}
\makeatletter
\def\maxwidth{\ifdim\Gin@nat@width>\linewidth\linewidth\else\Gin@nat@width\fi}
\def\maxheight{\ifdim\Gin@nat@height>\textheight\textheight\else\Gin@nat@height\fi}
\makeatother
% Scale images if necessary, so that they will not overflow the page
% margins by default, and it is still possible to overwrite the defaults
% using explicit options in \includegraphics[width, height, ...]{}
\setkeys{Gin}{width=\maxwidth,height=\maxheight,keepaspectratio}
% Set default figure placement to htbp
\makeatletter
\def\fps@figure{htbp}
\makeatother
\setlength{\emergencystretch}{3em} % prevent overfull lines
\providecommand{\tightlist}{%
  \setlength{\itemsep}{0pt}\setlength{\parskip}{0pt}}
\setcounter{secnumdepth}{5}

\title{Populating the study package}
\author{Jenna M. Reps}
\date{2020-03-19}

\begin{document}
\maketitle

{
\setcounter{tocdepth}{2}
\tableofcontents
}
\hypertarget{introduction}{%
\section{Introduction}\label{introduction}}

This vignette describes how one can populate the
SkeletonExistingModelStudy pacakge with the target cohort, outcome
cohorts and model settings.

First make sure to open the Skeleton R project in R studio, this can be
done by finding the SkeletonExistingModelStudy.Rproj file in the folder.
Once the package project is opened in R studio there are 3 steps that
must be followed:

\begin{enumerate}
\def\labelenumi{\arabic{enumi}.}
\tightlist
\item
  Run the function: populatePackage (found in extras/populatePackage.R
  on line 51) to add all cohorts and settings into the study package
\item
  Build the study package
\item
  Run the study package execute function
\end{enumerate}

\hypertarget{step-1-populate-skeleton-settings}{%
\subsection{Step 1: Populate skeleton
settings}\label{step-1-populate-skeleton-settings}}

All the settings can be added to the study package by using the function
`populatePackage()' that is found in extras/populatePackage.R.

To add the function to your environment, make sure the package R project
is open in R studio and run:

\begin{Shaded}
\begin{Highlighting}[]
\KeywordTok{source}\NormalTok{(}\StringTok{'./extras/populatePackage.R'}\NormalTok{)}
\end{Highlighting}
\end{Shaded}

This will make the function 'populatePackage()'available to use within
your R session.

The `populatePackage()' function requires users to specify:

\begin{itemize}
\tightlist
\item
  targetCohortId - The ATLAS id for the target cohort
\item
  targetCohortName - A string with a sharable name for the target cohort
\item
  outcomeId - The ATLAS id for the outcome cohort
\item
  outcomeName - A string with a sharable name for the outcome cohort
\item
  standardCovariates - A data.frame with the columns: covariateId (for
  standard fetures using FeatureExtraction), covariateName and points to
  assign points for standard covariates
\item
  baseUrl - The url for the ATLAS webapi (this will be used to extract
  the ATLAS cohorts)
\item
  atlasIds - an integer or vector of integers specifying the atlas
  cohort Ids that are used by the custom cohort covariates
\item
  atlasNames - a string or vector of strings specifying the names of the
  atlas ids (must be the same length as atlasIds)
\item
  startDays - a negative integer or vector of negative integers
  specifying the days relative to index to start looking for the patient
  being in the covariate cohort
\item
  endDays - a negative integer (or zero) or vector of negative integers
  (or zero) specifying the days relative to index to stop looking for
  the patient being in the covariate cohort
\item
  points - a double or vector of doubles specifying the points
  corresponding to each variable
\end{itemize}

For example, to create two custom cohort covariates into the package I
can run:

\begin{Shaded}
\begin{Highlighting}[]
\KeywordTok{populatePackage}\NormalTok{(}\DataTypeTok{targetCohortId =} \DecValTok{10845}\NormalTok{,}
                \DataTypeTok{targetCohortName =} \StringTok{'neg mamo'}\NormalTok{,}
                \DataTypeTok{outcomeId =} \DecValTok{10082}\NormalTok{,}
                \DataTypeTok{outcomeName =} \StringTok{'breast cancer'}\NormalTok{,}
                \DataTypeTok{standardCovariates =} \KeywordTok{data.frame}\NormalTok{(}\DataTypeTok{covariateId =} \KeywordTok{c}\NormalTok{(}\DecValTok{0003}\NormalTok{, }\DecValTok{1003}\NormalTok{,}
                                                                \DecValTok{2003}\NormalTok{, }\DecValTok{3003}\NormalTok{,}
                                                                \DecValTok{4003}\NormalTok{, }\DecValTok{5003}\NormalTok{,}
                                                                \DecValTok{6003}\NormalTok{, }\DecValTok{7003}\NormalTok{,}
                                                                \DecValTok{8003}\NormalTok{, }\DecValTok{9003}\NormalTok{,}
                                                                \DecValTok{10003}\NormalTok{, }\DecValTok{11003}\NormalTok{,}
                                                                \DecValTok{12003}\NormalTok{, }\DecValTok{13003}\NormalTok{,}
                                                                \DecValTok{14003}\NormalTok{, }\DecValTok{15003}\NormalTok{,}
                                                                \DecValTok{16003}\NormalTok{, }\DecValTok{17003}\NormalTok{,}
                                                                \DecValTok{8507001}\NormalTok{),}
                                                \DataTypeTok{covariateName =} \KeywordTok{c}\NormalTok{(}\StringTok{'Age 0-4'}\NormalTok{, }\StringTok{'Age 5-9'}\NormalTok{,}
                                                                  \StringTok{'Age 10-14'}\NormalTok{, }\StringTok{'Age 15-19'}\NormalTok{,}
                                                                  \StringTok{'Age 20-24'}\NormalTok{, }\StringTok{'Age 25-30'}\NormalTok{,}
                                                                  \StringTok{'Age 30-34'}\NormalTok{, }\StringTok{'Age 35-40'}\NormalTok{,}
                                                                  \StringTok{'Age 40-44'}\NormalTok{, }\StringTok{'Age 45-50'}\NormalTok{,}
                                                                  \StringTok{'Age 50-54'}\NormalTok{, }\StringTok{'Age 55-60'}\NormalTok{,}
                                                                  \StringTok{'Age 60-64'}\NormalTok{, }\StringTok{'Age 65-70'}\NormalTok{,}
                                                                  \StringTok{'Age 70-74'}\NormalTok{, }\StringTok{'Age 75-80'}\NormalTok{,}
                                                                  \StringTok{'Age 80-84'}\NormalTok{, }\StringTok{'Age 85-90'}\NormalTok{,}
                                                                  \StringTok{'Male'}\NormalTok{), }
                                                \DataTypeTok{points =} \KeywordTok{c}\NormalTok{(}\KeywordTok{rep}\NormalTok{(}\DecValTok{0}\NormalTok{,}\DecValTok{19}\NormalTok{))),}
                \DataTypeTok{baseUrl =} \StringTok{'https://yourWebAPI'}\NormalTok{,}
                \DataTypeTok{atlasCovariateIds =} \KeywordTok{c}\NormalTok{(}\DecValTok{14709}\NormalTok{,}\DecValTok{14709}\NormalTok{, }\DecValTok{14710}\NormalTok{),}
                \DataTypeTok{atlasCovariateNames =} \KeywordTok{c}\NormalTok{(}\StringTok{'smoking anytime'}\NormalTok{, }\StringTok{'smoking recent'}\NormalTok{, }\StringTok{'traumatic brain injury'}\NormalTok{),}
                \DataTypeTok{startDays =} \KeywordTok{c}\NormalTok{(}\OperatorTok{-}\DecValTok{999}\NormalTok{,}\OperatorTok{-}\DecValTok{30}\NormalTok{,}\OperatorTok{-}\DecValTok{999}\NormalTok{),}
                \DataTypeTok{endDays =}  \KeywordTok{c}\NormalTok{(}\DecValTok{0}\NormalTok{,}\DecValTok{0}\NormalTok{,}\DecValTok{0}\NormalTok{),}
                \DataTypeTok{points =} \KeywordTok{c}\NormalTok{(}\DecValTok{1}\NormalTok{,}\DecValTok{2}\NormalTok{,}\DecValTok{1}\NormalTok{))}
\end{Highlighting}
\end{Shaded}

The code above extracts the target and outcome cohorts and two ATLAS
cohort (14709, 14710) to create three covariates:

\begin{itemize}
\tightlist
\item
  covariate 1: The ATLAS cohort with the id of 14709 named `smoking
  anytime' looks for patients who have a smoking anytime
  cohort\_start\_date between (index date-999 days) and (index date).
  E.g., If a patient is in the smoking anytime cohort 50 days before the
  index date then they will have a value of 1 for the custom covariate.
  If they are not in the smoking anytime cohort between 999 days before
  index and the day of index then they will have a value of 0 for the
  custom covariate.
\item
  covariate 2: The ATLAS cohort with the id of 14709 named `smoking
  recent' looks for patients who have a smoking recent
  cohort\_start\_date between (index date-30 days) and (index date).
  E.g., If a patient is in the smoking recent cohort 20 days before the
  index date then they will have a value of 1 for the custom covariate.
  If they are not in the smoking recent cohort between 30 days before
  index and the day of index then they will have a value of 0 for the
  custom covariate.
\item
  covariate 3: The ATLAS cohort with the id of 14710 named `traumatic
  brain injury' looks for patients who have a traumatic brain injury
  cohort\_start\_date between (index date-999 days) and (index date).
  E.g., If a patient is in the traumatic brain injury cohort 200 days
  before the index date then they will have a value of 1 for the custom
  covariate. If they are not in the traumatic brain injury cohort before
  index then they will have a value of 0 for the custom covariate.
\end{itemize}

It also creates three csv files in the inst/settings directory named:

\begin{itemize}
\tightlist
\item
  CohortsToCreate.csv - specifying the target and outcome cohorts
\item
  CustomCovariates.csv - specifying the custom covariates
\item
  SimpleModel.csv - settings specifying the simple prediction model
\end{itemize}

\hypertarget{step-2-build-the-study-package}{%
\subsection{Step 2: Build the study
package}\label{step-2-build-the-study-package}}

Aftering adding the settings into the package, you now need to build the
package. Use the standard process (in R studio press the `Build' tab in
the top right corner and then select the `Install and Restart' button)
to build the study package so an R library is created.

\hypertarget{step-3-execute-the-study-to-validate-an-existing-model}{%
\subsection{Step 3: Execute the study to validate an existing
model}\label{step-3-execute-the-study-to-validate-an-existing-model}}

\begin{Shaded}
\begin{Highlighting}[]
  \KeywordTok{library}\NormalTok{(SkeletonExistingModelStudy)}
  \KeywordTok{options}\NormalTok{(}\DataTypeTok{fftempdir =} \StringTok{"location with space to save big data"}\NormalTok{)}
  
  \CommentTok{# The folder where the study intermediate and result files will be written:}
\NormalTok{  outputFolder <-}\StringTok{ "./SkeletonExistingModelStudyResults"}
  
  \CommentTok{# Details for connecting to the server:}
\NormalTok{  dbms <-}\StringTok{ "you dbms"}
\NormalTok{  user <-}\StringTok{ 'your username'}
\NormalTok{  pw <-}\StringTok{ 'your password'}
\NormalTok{  server <-}\StringTok{ 'your server'}
\NormalTok{  port <-}\StringTok{ 'your port'}
  
\NormalTok{  connectionDetails <-}\StringTok{ }\NormalTok{DatabaseConnector}\OperatorTok{::}\KeywordTok{createConnectionDetails}\NormalTok{(}\DataTypeTok{dbms =}\NormalTok{ dbms,}
                                                                  \DataTypeTok{server =}\NormalTok{ server,}
                                                                  \DataTypeTok{user =}\NormalTok{ user,}
                                                                  \DataTypeTok{password =}\NormalTok{ pw,}
                                                                  \DataTypeTok{port =}\NormalTok{ port)}
  
  \CommentTok{# Add the database containing the OMOP CDM data}
\NormalTok{  cdmDatabaseSchema <-}\StringTok{ 'cdm database schema'}
  \CommentTok{# Add a database with read/write access as this is where the cohorts will be generated}
\NormalTok{  cohortDatabaseSchema <-}\StringTok{ 'work database schema'}
  
\NormalTok{  oracleTempSchema <-}\StringTok{ }\OtherTok{NULL}
  
  \CommentTok{# table name where the cohorts will be generated}
\NormalTok{  cohortTable <-}\StringTok{ 'SkeletonPredictionStudyCohort'}
  
  \CommentTok{# TAR settings}
\NormalTok{  sampleSize <-}\StringTok{ }\OtherTok{NULL}
\NormalTok{  riskWindowStart <-}\StringTok{ }\DecValTok{1}
\NormalTok{  startAnchor <-}\StringTok{ 'cohort start'}
\NormalTok{  riskWindowEnd <-}\StringTok{ }\DecValTok{365}
\NormalTok{  endAnchor <-}\StringTok{ 'cohort start'}
\NormalTok{  firstExposureOnly <-}\StringTok{ }\NormalTok{F}
\NormalTok{  removeSubjectsWithPriorOutcome <-}\StringTok{ }\NormalTok{F}
\NormalTok{  priorOutcomeLookback <-}\StringTok{ }\DecValTok{99999}
\NormalTok{  requireTimeAtRisk <-}\StringTok{ }\NormalTok{F}
\NormalTok{  minTimeAtRisk <-}\StringTok{ }\DecValTok{1}
\NormalTok{  includeAllOutcomes <-}\StringTok{ }\NormalTok{T}
  
  
  \CommentTok{#=======================}
  
\NormalTok{  standardCovariates <-}\StringTok{ }\NormalTok{FeatureExtraction}\OperatorTok{::}\KeywordTok{createCovariateSettings}\NormalTok{(}\DataTypeTok{useDemographicsAgeGroup =}\NormalTok{ T, }\DataTypeTok{useDemographicsGender =}\NormalTok{ T)}
  
\NormalTok{  SkeletonExistingModelStudy}\OperatorTok{::}\KeywordTok{execute}\NormalTok{(}\DataTypeTok{connectionDetails =}\NormalTok{ connectionDetails,}
                                      \DataTypeTok{cdmDatabaseSchema =}\NormalTok{ cdmDatabaseSchema,}
                                      \DataTypeTok{cdmDatabaseName =}\NormalTok{ cdmDatabaseName,}
                                      \DataTypeTok{cohortDatabaseSchema =}\NormalTok{ cohortDatabaseSchema,}
                                      \DataTypeTok{cohortTable =}\NormalTok{ cohortTable,}
                                      \DataTypeTok{sampleSize =}\NormalTok{ sampleSize,}
                                      \DataTypeTok{riskWindowStart =}\NormalTok{ riskWindowStart,}
                                      \DataTypeTok{startAnchor =}\NormalTok{ startAnchor,}
                                      \DataTypeTok{riskWindowEnd =}\NormalTok{ riskWindowEnd,}
                                      \DataTypeTok{endAnchor =}\NormalTok{ endAnchor,}
                                      \DataTypeTok{firstExposureOnly =}\NormalTok{ firstExposureOnly,}
                                      \DataTypeTok{removeSubjectsWithPriorOutcome =}\NormalTok{ removeSubjectsWithPriorOutcome,}
                                      \DataTypeTok{priorOutcomeLookback =}\NormalTok{ priorOutcomeLookback,}
                                      \DataTypeTok{requireTimeAtRisk =}\NormalTok{ requireTimeAtRisk,}
                                      \DataTypeTok{minTimeAtRisk =}\NormalTok{ minTimeAtRisk,}
                                      \DataTypeTok{includeAllOutcomes =}\NormalTok{ includeAllOutcomes,}
                                      \DataTypeTok{standardCovariates =}\NormalTok{ standardCovariates,}
                                      \DataTypeTok{outputFolder =}\NormalTok{ outputFolder,}
                                      \DataTypeTok{createCohorts =}\NormalTok{ T,}
                                      \DataTypeTok{runAnalyses =}\NormalTok{ T,}
                                      \DataTypeTok{viewShiny =}\NormalTok{ T,}
                                      \DataTypeTok{packageResults =}\NormalTok{ F,}
                                      \DataTypeTok{minCellCount=} \DecValTok{5}\NormalTok{,}
                                      \DataTypeTok{verbosity =} \StringTok{"INFO"}\NormalTok{,}
                                      \DataTypeTok{cdmVersion =} \DecValTok{5}\NormalTok{)}
\ErrorTok{)}
\end{Highlighting}
\end{Shaded}

\end{document}
